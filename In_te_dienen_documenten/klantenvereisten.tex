\documentclass[a4paper,kulak]{kulakarticle} %options: kul or kulak (default)

\usepackage[utf8]{inputenc}
\usepackage[dutch]{babel}

\date{Academiejaar 2020 -- 2021}
\address{
  Bachelor in ingenieurswetenschappen \\
  Probleemoplossen en ontwerpen 2 \\
  Benjamin Maveau, Kevin Truyaert}
\title{Samenvatting klantenvereisten}
\author{Team 1: Safety First}


\begin{document}

\maketitle

\section{Concept}

Bouwen van een gemotoriseerd voertuig dat autonoom kan rondrijden in een Smart City. Het voertuig moet straten volgen via een volglijn en andere voertuigen en verkeerslichten bij kruispunten kunnen detecteren. Bij een rood licht moet het voertuig stoppen aan de stopstreep. Het voertuig mag niet botsen op andere voertuigen. Het te volgen traject wordt vooraf geïmplementeerd. De maximale kostprijs van dit voertuig is 3500 virtuele eenheden. De relevante gegevens van dit voertuig kunnen uitgelezen worden vanop afstand. Een manuele override moet ook mogelijk zijn, zij het uitvoeren van een noodstop, zij de besturing overnemen.



\end{document}
