\documentclass[a4paper,kulak]{kulakarticle} %options: kul or kulak (default)

\usepackage[utf8]{inputenc}
\usepackage[dutch]{babel}

\date{\today}
\address{
  Ingenieurswetenschappen \\
  P\&O2 \\
  Benjamin Maveau, Kevin Truyaert}
\title{Vergaderingsverslag: Week 8}
\author{Team 1}


\begin{document}

\maketitle

\section{Verslag begin sessie}

\subsection{Bespreking verslag vorige vergadering}

\subsection{Evaluatie activiteiten}

\subsection{Planning}
\begin{itemize}
	\item tussentijdse presentatie afwerken (Camille)
	\item Afstandssensor implementatie evalueren met nieuwe sensor (Otto) 
	\item Lijnsensor implementatie uitbreiden en verder monteren van wagentje (Ruben)
	\item Serverconnectie LabVIEW \& PyCharm (Emiel)
	\item Ondersteuning bij programmeren (Otto)
\end{itemize}

\subsection{Groepsfunctioneren}

\subsection{Rondvraag}

\subsection{Varia}




\section{Verslag einde sessie}

\subsection{Bespreking verslag vorige vergadering}

\subsection{Evaluatie activiteiten}
Na overleg met Benjamin over de trage motoren zijn, is zijn advies duidelijk, we zullen nieuwe motoren met meer torque moeten bestellen, de enige optie is 100:1. De afstandssensor werkt goed, de kleursensor heeft zware vertraging opgelopen door de installatie van de packages, dit is nog niet klaar. Tussentijdse presentatie is ook af ondertussen. Het elektrische aansluitingsdiagram in deftig formaat is bijna af, zou volgende sessie moeten af kunnen. De assemblage schiet wat minder op dan gehoopt. 

\subsection{Planning}

\subsection{Groepsfunctioneren}

\subsection{Rondvraag}

\subsection{Varia}

\end{document}
