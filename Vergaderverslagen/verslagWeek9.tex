\documentclass[a4paper,kulak]{kulakarticle} %options: kul or kulak (default)

\usepackage[utf8]{inputenc}
\usepackage[dutch]{babel}

\date{\today}
\address{
  Ingenieurswetenschappen \\
  P\&O2 \\
  Benjamin Maveau, Kevin Truyaert}
\title{Vergaderingsverslag: Week 9}
\author{Team 1}


\begin{document}

\maketitle

\section{Verslag begin sessie}
\subsection{Bespreking verslag vorige vergadering}
Er is nog heel wat werk te doen in een nogal beperkte tijd, we zullen efficiënt te werk moeten gaan. 
\subsection{Evaluatie activiteiten}

\subsection{Planning}
\subsubsection*{Op campus}
\begin{itemize}
	\item Motoren vervangen op chassis
	\item Lijnvolgalgoritme testen
	\item Solderen
	\item Kleursensor bevestigen
	\item Powerbank \& Raspberry bevestigen op chassis
	\item kabels wat managen met spanbandjes
	\item Motorshield instellen ipv. Dual Drive
\end{itemize}

\subsubsection*{Thuis}
\begin{itemize}
	\item Eindverslag typen
	\item Hoofdprogramma verder afwerken
	\item Elektrisch circuit afwerken
	\item functie Kruispunt() afwerken
	\item manier vinden om kleursensor packages te installeren die al op rasberry staan
\end{itemize}
\subsection{Groepsfunctioneren}
Om meer mensen fysiek aan de robot te kunnen werken op campus dachten we een soort rotatiesysteem uit te werken. 
\subsection{Rondvraag}
Welke extra momenten gaan we naar campus? Maandagmiddag 26 april (Otto, Staf), woensdagnamiddag 28 april (Ruben, Emiel), maandagnamiddag 3 mei (?, ?) woensdagnamiddag 5 mei(Camille, Ruben).

\subsection{Varia}




\section{Verslag einde sessie}

\subsection{Bespreking verslag vorige vergadering}

\subsection{Evaluatie activiteiten}
\begin{itemize}
	\item Nieuwe motoren geïnstalleerd, Raspberry gemonteerd op chassis, powerbank ook bevestigd
	\item Lijnalgoritme werkt bijzonder goed en is definitief
	\item Eerste versie van manuele override werkt
	\item Kruispunt.py en kleurensensor werden geüpdatet maar moeten nog getest worden.
	\item De meeste opmerkingen zijn al toegepast op het eindverslag.
\end{itemize}
\subsection{Planning}

\subsubsection*{Otto en Staf maandag}
\begin{itemize}
	\item solderen van motorshield met diodes
	\item kleursensor uitlezen
\end{itemize}

\subsubsection*{Emiel en Ruben woensdag}
\begin{itemize}
	\item 'achteruit' en perfectioneren van pwm
	
\end{itemize}

\subsubsection*{Camille}
Verder werken aan het eindverslag, het grootste deel zal af zijn tegen volgende week vrijdag.

\subsection{Groepsfunctioneren}

\subsection{Rondvraag}

\subsection{Varia}
Wat valt er precies te solderen? Volgens Ruben enkel motorshield, volgens de rest alles. We hebben een vraag gesteld in de discord.

\end{document}
