\documentclass[a4paper,kulak]{kulakarticle} %options: kul or kulak (default)

\usepackage[utf8]{inputenc}
\usepackage[dutch]{babel}

\date{\today}
\address{
  Ingenieurswetenschappen \\
  P\&O2 \\
  Benjamin Maveau, Kevin Truyaert}
\title{Vergaderingsverslag: Week 6}
\author{Team 1: Safety First}


\begin{document}

\maketitle

\section{Verslag begin sessie}
\subsection{Bespreking verslag vorige vergadering}
\subsection{Evaluatie activiteiten}
\subsection{Planning}
We hebben nu onze componenten in het echt eens gezien. We zullen het CAD-model van de robot moeten updaten, aangezien de RPI in een behuizing moet blijven

Hieronder de planning voor vandaag:
\begin{itemize}
	\item PID afwerken, componenten nakijken (Ruben)
	\item Tussentijds verslag verder afwerken(Camille \& Otto)
	\item CAD-model updaten, meehelpen programmeren (Emiel)
	\item Globale programmeerstructuur beginnen implementeren (Staf)
\end{itemize}

\subsection{Groepsfunctioneren}
\subsection{Rondvraag}
\subsection{Varia}




\section{Verslag einde sessie}

\subsection{Bespreking verslag vorige vergadering}

\subsection{Evaluatie activiteiten}

\subsection{Planning}
Iedereen gaat de professionele editie van PyCharm moeten downloaden en doorloopt de stapjes in SSH-module op Toledo. 
\subsection{Groepsfunctioneren}

\subsection{Rondvraag}

\subsection{Varia}

\end{document}
