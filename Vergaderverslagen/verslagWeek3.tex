\documentclass[a4paper,kulak]{kulakarticle} %options: kul or kulak (default)

\usepackage[utf8]{inputenc}
\usepackage[dutch]{babel}

\date{\today}
\address{
  Ingenieurswetenschappen \\
  P\&O2 \\
  Benjamin Maveau, Kevin Truyaert}
\title{Vergaderingsverslag: Week 3}
\author{Team 1}


\begin{document}

\maketitle

\section{Verslag begin sessie}

\subsection{Bespreking verslag vorige vergadering}
Nieuwe voorkeur voor microcontroller de myRIO, voor gemak van analoge sensoren en de samenhang met LabVIEW. We stellen de eerste vier plaatsen als niet prioritair. 

\subsection{Evaluatie activiteiten}
Wat betreft documenten, meer vooraf plannen en beter vastleggen wie wat doet.
Wat meer communicatie via Discord, maar het gemak van Messenger is te groot, we zullen niet volledig overschakelen van het ene op het andere platform. 


\subsection{Planning}

\begin{itemize}
\item Deze morgen: Onderdelen concreet bespreken, de 2 scenario's (MyRIO of Raspberry). Mooi uitschrijven in Excel. 

\item 15u: bieding

\item Na bieding: begin modelleren, info componenten opzoeken, evt. programmastructuren opstellen.

\end{itemize}

\subsection{Groepsfunctioneren}
Over het algemeen moeten we meer communiceren met elkaar. Momenteel delen we alleen mee als er iets afgewerkt is of gedaan. We moeten ook concreter taken in plannen en duidelijk vermelden wie welke taak op zich neemt. Op die manier zal iedereen gelijkwaardige bijdragen leveren en weet iedereen beter wat er van hem verwacht wordt.

\subsection{Rondvraag}
/
\subsection{Varia}
/



\section{Verslag einde sessie}

\subsection{Bespreking verslag vorige vergadering}
Positie op bieding = laatste plek. Vrij grote onzekerheid wat betreft componenten en platform dat we zullen moeten gebruiken(wss. Raspberry Pi). Een globaal model kunnen we wel al schetsen.

\subsection{Evaluatie activiteiten}
Custom CAD-model van het chassis vordert goed. Het begrip over de I/O-pins van de Raspberry Pi is al veel verbeterd.

\subsection{Planning}
/
\subsection{Groepsfunctioneren}
/
\subsection{Rondvraag}
/
\subsection{Varia}
/
\end{document}
