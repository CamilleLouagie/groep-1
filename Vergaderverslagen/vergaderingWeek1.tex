\documentclass[a4paper,kulak]{kulakarticle} %options: kul or kulak (default)

\usepackage[utf8]{inputenc}
\usepackage[dutch]{babel}

\date{\today}
\address{
  Ingenieurswetenschappen \\
  P\&O2 \\
  Naam van de docent/begeleiders}
\title{Vergaderingsverslag: Week 1}
\author{Team Safety First}


\begin{document}

\maketitle

\section{Verslag begin sessie}

\subsection{Bespreking verslag vorige vergadering}
/
\subsection{Evaluatie activiteiten}
/
\subsection{Planning}
We moeten tegen 16u een voorlopig ontwerp klaar hebben.

We moeten ook een overzicht van de  taakstructuur, teamkalender, verantwoordelijkheidsstructuur en een Gantt chart opstellen. Dit moet ingediend worden tegen dinsdag 16 februari. Emiel neemt de taakstructuur voor zich, Camille maakt de teamkalender, Staf de verantwoordelijkheidsstructuur, Emiel maakt ook de Gantt chart.

Er moet een Github gemaakt worden, teamleider Camille doet dit. Otto stelt een voorlopig kostenplaatje op.

\subsection{Groepsfunctioneren}
/

\subsection{Rondvraag}
/

\subsection{Varia}
Overleaf gebruiken voor de te maken \LaTeX-documenten zodat iedereen van het team ze vrij kan bewerken.


\section{Verslag einde sessie}

\subsection{Bespreking verslag vorige vergadering}
Het voorlopig ontwerp is nog niet concreet genoeg, we moeten ons nog meer inlezen over de componenten. Dit moet echter snel gebeuren, tegen dat we moeten bieden.


\subsection{Evaluatie activiteiten}

\subsubsection{Documenten}

Samenvatting klantenvereisten, overzicht productspecificaties en de verantwoordelijkheidsstructuur zijn af. Taakkalender zo goed als klaar, nog nagelezen worden. De taakstructuur moet enkel nog het tweede deel aangevuld worden. De Gantt chart moet ook nog afwerkt worden.

Github is opgestart en operationeel.  De voorlopige kostenraming op basis van de voorlopige onderdelenlijst is af, zeer waarschijnlijk zal deze aangepast.

 
\subsubsection{Ontwerp}
Grootste voorkeur voor de individuele touch: 3D printen van het frame.

Voorlopige onderdelenlijst:
\begin{itemize}
	\item Breadboard: onbeslist
	\item Wielen: Wiel 42x19mm
	\item Ball Caster: staat vast wegens 1 mogelijkheid
	\item Motor: onbeslist
	\item Een set Gearmotor Brackets
	\item Motor Shield: Allebei, wegens geen eerdere ervaring
	\item Kleursensor: staat vast wegens 1 mogelijkheid
	\item Afstandssensor Digitaal, want Raspberry kan enkel met digitaal overweg 
	\item Reflectiesensor Digitaal, zie hier boven
	\item Microcontroller: Raspberry Pi
	\item Batterij powerbank
	\item Arm voor de kleursensor
	
	
\end{itemize}

Iedereen leest zich nog meer in over deze componenten.

\subsubsection{Verantwoordelijkheden}
\begin{itemize}
	\item Camille: teamleider
	\item Ruben: Eindverantwoordelijke Constructie
	\item Staf: Notulist
	\item Otto: Penningmeester
	\item Emiel: Eindverantwoordelijke Programmeren
\end{itemize}


\subsection{Planning}
Gantt chart en taakstructuur moeten we afwerken tegen dinsdag. Taakstructuur kunnen we elk een voorbeeld geven in het discord tekstkanaal. We lezen allemaal ons allemaal nog wat in

\subsection{Groepsfunctioneren}
Niet echt iets op aan te merken iedereen deed zijn werk goed.

\subsection{Rondvraag}
/

\subsection{Varia}
/

\end{document}




