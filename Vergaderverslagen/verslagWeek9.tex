\documentclass[a4paper,kulak]{kulakarticle} %options: kul or kulak (default)

\usepackage[utf8]{inputenc}
\usepackage[dutch]{babel}

\date{\today}
\address{
  Ingenieurswetenschappen \\
  P\&O2 \\
  Benjamin Maveau, Kevin Truyaert}
\title{Vergaderingsverslag: Week 9}
\author{Team 1}


\begin{document}

\maketitle

\section{Verslag begin sessie}
\subsection{Bespreking verslag vorige vergadering}
Er is nog heel wat werk te doen in een nogal beperkte tijd, we zullen efficiënt te werk moeten gaan. 
\subsection{Evaluatie activiteiten}

\subsection{Planning}
\subsubsection*{Op campus}
\begin{itemize}
	\item Motoren vervangen op chassis
	\item Lijnvolgalgoritme testen
	\item Solderen
	\item Kleursensor bevestigen
	\item Powerbank \& Raspberry bevestigen op chassis
	\item kabels wat managen met spanbandjes
\end{itemize}

\subsubsection*{Thuis}
\begin{itemize}
	\item Eindverslag typen
	\item Hoofdprogramma verder afwerken
	\item 
\end{itemize}
\subsection{Groepsfunctioneren}
Om meer mensen fysiek aan de robot te kunnen werken op campus dachten we een soort rotatiesysteem uit te werken. 
\subsection{Rondvraag}
Welke extra momenten gaan we naar campus? Maandagmiddag 26 april (Otto, Staf), woensdag 28 april (Ruben, Emile/Staf), woensdag 5 mei(Camille, ?).

\subsection{Varia}




\section{Verslag einde sessie}

\subsection{Bespreking verslag vorige vergadering}

\subsection{Evaluatie activiteiten}

\subsection{Planning}

\subsection{Groepsfunctioneren}

\subsection{Rondvraag}

\subsection{Varia}
Wat valt er precies te solderen?

\end{document}
