\documentclass[a4paper,kulak]{kulakarticle} %options: kul or kulak (default)

\usepackage[utf8]{inputenc}
\usepackage[dutch]{babel}

\date{Academiejaar 2020 -- 2021}
\address{
  Bachelor in ingenieurswetenschappen \\
  Probleemoplossen en ontwerpen, deel 2 \\
  Benjamin Maveau, Kevin Truyaert}
\title{Klantenvereisten}
\author{Team 1: Safety First}


\begin{document}

\maketitle

\section{Concept}

De klant wenst een miniatuur robotwagen die autonoom kan rondrijden volgens een voorgeprogrammeerde route in een modelstad. Hierbij volgt de robotwagen straten via een volglijn. Daarnaast kan het voertuig voorliggers of obstakels detecteren en stoppen indien nodig, om botsing te vermijden. De klant wenst ook dat het voertuig verkeerslichten kan interpreteren bij kruispunten. Indien het verkeerslicht rood is, stopt de robotwagen bij de stopstreep. Indien het verkeerslicht groen is, rijdt de robotwagen door.  Het te volgen traject wordt vooraf geïmplementeerd. De robotwagen kan afslaan op een kruispunt indien dit zo in de beschrijving stond. De maximale kostprijs van dit voertuig is 3500 virtuele eenheden. De klant wenst een grafische interface waarmee de relevante gegevens van dit voertuig kunnen uitgelezen worden vanop afstand. Een manuele overname van de robotwagen moet ook mogelijk zijn, zij het uitvoeren van een noodstop, zij de besturing overnemen.



\end{document}
