\documentclass[a4paper,kulak]{kulakarticle} %options: kul or kulak (default)

\usepackage[utf8]{inputenc}
\usepackage[dutch]{babel}

\date{\today}
\address{
  Ingenieurswetenschappen \\
  P\&O2 \\
  Benjamin Maveau, Kevin Truyaert}
\title{Vergaderingsverslag: Week 7}
\author{Team 1}


\begin{document}

\maketitle

\section{Verslag begin sessie}

\subsection{Bespreking verslag vorige vergadering}
We komen nu met 2 man op campus, Staf en Ruben. Dit omdat Ruben verantwoordelijke is voor constructie en omdat Staf een voor de moment meer geprogrammeerd heeft dan de verantwoordelijke voor programmeren Emiel.
\subsection{Planning}

\subsection{Evaluatie activiteiten}

\begin{itemize}
	\item tussentijds verslag controleren op kleine foutjes (Camille)
	\item Beamerpresentatie maken (Otto)
	\item LabVIEW serverimplementatie (Emiel)
	\item Gemaakte implementaties testen op fouten (Ruben)
	\item Schakelingen maken en beginnen aan fysieke assemblage (Staf)
\end{itemize}

\subsection{Groepsfunctioneren}

\subsection{Rondvraag}

\subsection{Varia}




\section{Verslag einde sessie}

\subsection{Bespreking verslag vorige vergadering}


\subsection{Evaluatie activiteiten}
We gaan een nieuwe bestelling plaatsen: 2 afstandsbussen van 15mm, een MakerBeam van 100mm, een Makerbeam van 40mm, een Makerbeam van 60mm, 2x Male headers van 10, 4 wire to board sockets. Dit samen komt op 100 credits. Het tussentijds verslag is op tijd afgewerkt, de Beamer-presentatie is echter nog niet af. De motoren kunnen draaien, de reflectiesensor werkt, de afstandssensor geeft waarden terug maar inconsistent, we moeten Benjamin een mail zenden om een nieuwe sensor te komen geven volgende week. De kleursensor hebben we nog niet kunnen testen. Qua constructie komt het er ook vooral op aan om de componenten mooi te ordenen op ons chassis. Het blijkt ook dat de achterkant van ons robotvoertuig nogal instabiel is door de zware powerbank langs achteren op de ballcaster te positioneren. We zouden eventueel proberen de Raspberry wat op te schuiven en de powerbank ernaast te leggen.


\subsection{Planning}
Emiel maakt een blok voor de afstand van de ballcaster tov van het chassis te vergroten, via Solid Edge en zal dit laten 3D-printen tegen volgende sessie.

\subsection{Groepsfunctioneren}

\subsection{Rondvraag}

\subsection{Varia}

\end{document}
